\documentclass{sprz}

\addbibresource{bibliography.bib}

\studfield{Informatyka}
\studtype{Zaoczne}
\title{Praca inżynierska o tym i owym}
\engtitle{Something about this and that}
\supervisor{dr Puźniakowski Tadeusz}
\author{Nowak Jan}{s1234}
\author{Doe John}{s1222}
\author{Alice Bob}{s1299}
\date{\today}
\nabstract{To jest streszczenie niezwykle innowacyjnej pracy inżynierskiej w której przeplatają się elementy paplaniny z wygenerowanym tekstem pseudonaukowym rozpoznawanym jako dezinformacja.}



\begin{document}

\maketitle

\tableofcontents

\chapter{Informacje wstępne}



\section{O projekcie}\label{ch:wstep}

To jest przykładowy szablon pracy inżynierskiej na PJATK.

\chapter{Przykładowe elementy}

\section{Listingi}

Jak załączać kod źródłowy jest pokazane na listingu~\ref{lst:helloworld}


\begin{lstlisting}[language=c,caption={Przykładowy witaj w świecie}, label={lst:helloworld}]
printf("hello");
\end{lstlisting}

\section{Obrazki}

Na ilustracji~\ref{img:pjatklogo} widzimy oficjalne logo PJATK.

\begin{figure}[h]
    \centering
    \includegraphics[width=0.5\textwidth]{sprz/pjatk}
    \caption{Logo PJATK załączone jako obrazek}
    \label{img:pjatklogo}
\end{figure}

albo dla wygody jako makro tak jak na obrazku~\ref{img:pjatklogo2}

\putimage{Obrazek załączony za pomocą makra}{sprz/pjatk}{img:pjatklogo2}{0.5\textwidth}

\chapter{Karty udziałowca}


\begin{stakeholder}[label={tab:stakeholder:someholder},caption={Przykładowy opis udzialowca}]
    \id{jednoznaczny symbol np. UOB 01, UOB 02 ... dla udziałowców ożywionych bezpośrednich, UNP 01... dla nieożywionych pośrednich itd.}
    \name{nazwa udziałowca}
    \descr{opis udziałowca}
    \type{ożywiony/nieożywiony, bezpośredni/pośredni}
    \viewpoint{z jakiej perspektywy patrzy udziałowiec np. technicznej, ekonomicznej, operatora systemu itp.}
    \limitations{ograniczenia udziałowca np. administrator nie powinien specyfikować wymagań finansowych}
    \requ{tu tylko symbole wymagań wyspecyfikowanych w rozdziale 3}
\end{stakeholder}

\chapter{Wymagania wszelakie}

Na tabeli \ref{tab:requirements:general} pokazano jak można definiować wymagania ogólne lub dziedzinowe.

\begin{requirementstab}[label={tab:requirements:general},caption={Przykładowe wymaganie ogólne lub dziedzinowe}]
    \id{jednoznaczny symbol np. WO1, WO2 .. } 
    \priority{ważność wymagania, np. wg skali MoSCoW: M – must (musi być) S – should (powinno być) C – could (może być) W – won't (nie będzie – nie będzie implementowane w danym wydaniu, ale może być rozpatrzone w przyszłości )}
    \name{krótki opis} 
    \descr{opis szczegółowy, należy dążyć do tego, żeby wszystkie znane na ten moment szczegóły wymagania zostały wydobyte i wyspecyfikowane} 
    \sholder{nazwa udziałowca, który podał wymaganie} 
    \reqrelated{wymagania zależne i uszczegóławiające – odesłanie poprzez identyfikator} 
\end{requirementstab}

\printbibliography[title={Bibliografia}]




\end{document}