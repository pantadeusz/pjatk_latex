\documentclass{sprz}

\addbibresource{bibliography.bib}

\studfield{Informatyka}
\studtype{Zaoczne}
\title{Praca inżynierska o tym i owym}
\engtitle{Something about this and that}
\supervisor{dr Puźniakowski Tadeusz}
\author{Nowak Jan}{s1234}
\author{Doe John}{s1222}
\author{Alice Bob}{s1299}
\date{\today}
\nabstract{To jest streszczenie niezwykle innowacyjnej pracy inżynierskiej w której przeplatają się elementy paplaniny z wygenerowanym tekstem pseudonaukowym rozpoznawanym jako dezinformacja.}



\begin{document}

\maketitle

\tableofcontents

\chapter{Informacje wstępne}



\section{O projekcie}\label{ch:wstep}

To jest przykładowy szablon pracy inżynierskiej na PJATK.

\chapter{Przykładowe elementy}

\section{Listingi}

Jak załączać kod źródłowy jest pokazane na listingu~\ref{lst:helloworld}


\begin{lstlisting}[language=c,caption={Przykładowy witaj w świecie}, label={lst:helloworld}]
printf("hello");
\end{lstlisting}

\section{Obrazki}

Na ilustracji~\ref{img:pjatklogo} widzimy oficjalne logo PJATK.

\begin{figure}[h]
    \centering
    \includegraphics[width=0.5\textwidth]{sprz/pjatk}
    \caption{Logo PJATK załączone jako obrazek}
    \label{img:pjatklogo}
\end{figure}

albo dla wygody jako makro tak jak na obrazku~\ref{img:pjatklogo2}

\putimage{Obrazek załączony za pomocą makra}{sprz/pjatk}{img:pjatklogo2}{0.5\textwidth}

\chapter{Karty udziałowca}

% \begin{table}[h]
%     \centering
%     \begin{tabular}{|>{\columncolor{lightgray}}l|p{0.6\linewidth}|}
%         \hline
%         \rowcolor{lightgray}\multicolumn{2}{|c|}{\textbf{KARTA UDZIAŁOWCA}} \\
%         \hline
%         Identyfikator: & UOP 01 \\
%         Nazwa: & Opiekun projektu \\
%         Opis: & Osoba nadzorująca projekt \\
%         Typ udziałowca: & Ożywiony pośredni \\
%         Punkt widzenia: & Z perspektywy technicznej, opiekuna projektu, promotora \\
%         Ograniczenia: & Brak \\
%         Wymagania: & tu tylko symbole wymagań wyspecyfikowanych w rozdziale 3 \\
%         \hline
%     \end{tabular}
%     %\caption{Karta udziałowca}
%     \label{tab:karta_udzialowca}
% \end{table}

\stakeholder{UOP 01}{Zespół projektowy}{To ci co robią projekt}{Ożywiony bezpośredni}{Właściciel projektu, główny wykonawca}{Tu jakieś ograniczenia}{Można tu się odnieść do innego rozdziału dotyczącego implementacji, na przykład~\ref{ch:wstep}}{tab:sh:firstone}

\printbibliography[title={Bibliografia}]

\end{document}